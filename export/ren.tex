% Options for packages loaded elsewhere
\PassOptionsToPackage{unicode}{hyperref}
\PassOptionsToPackage{hyphens}{url}
%
\documentclass[
]{article}
\usepackage{lmodern}
\usepackage{amsmath}
\usepackage{ifxetex,ifluatex}
\ifnum 0\ifxetex 1\fi\ifluatex 1\fi=0 % if pdftex
  \usepackage[T1]{fontenc}
  \usepackage[utf8]{inputenc}
  \usepackage{textcomp} % provide euro and other symbols
  \usepackage{amssymb}
\else % if luatex or xetex
  \usepackage{unicode-math}
  \defaultfontfeatures{Scale=MatchLowercase}
  \defaultfontfeatures[\rmfamily]{Ligatures=TeX,Scale=1}
\fi
% Use upquote if available, for straight quotes in verbatim environments
\IfFileExists{upquote.sty}{\usepackage{upquote}}{}
\IfFileExists{microtype.sty}{% use microtype if available
  \usepackage[]{microtype}
  \UseMicrotypeSet[protrusion]{basicmath} % disable protrusion for tt fonts
}{}
\makeatletter
\@ifundefined{KOMAClassName}{% if non-KOMA class
  \IfFileExists{parskip.sty}{%
    \usepackage{parskip}
  }{% else
    \setlength{\parindent}{0pt}
    \setlength{\parskip}{6pt plus 2pt minus 1pt}}
}{% if KOMA class
  \KOMAoptions{parskip=half}}
\makeatother
\usepackage{xcolor}
\IfFileExists{xurl.sty}{\usepackage{xurl}}{} % add URL line breaks if available
\IfFileExists{bookmark.sty}{\usepackage{bookmark}}{\usepackage{hyperref}}
\hypersetup{
  pdftitle={named entity recognition},
  pdfauthor={Selim Ben Ismail},
  hidelinks,
  pdfcreator={LaTeX via pandoc}}
\urlstyle{same} % disable monospaced font for URLs
\usepackage[margin=1in]{geometry}
\usepackage{color}
\usepackage{fancyvrb}
\newcommand{\VerbBar}{|}
\newcommand{\VERB}{\Verb[commandchars=\\\{\}]}
\DefineVerbatimEnvironment{Highlighting}{Verbatim}{commandchars=\\\{\}}
% Add ',fontsize=\small' for more characters per line
\usepackage{framed}
\definecolor{shadecolor}{RGB}{248,248,248}
\newenvironment{Shaded}{\begin{snugshade}}{\end{snugshade}}
\newcommand{\AlertTok}[1]{\textcolor[rgb]{0.94,0.16,0.16}{#1}}
\newcommand{\AnnotationTok}[1]{\textcolor[rgb]{0.56,0.35,0.01}{\textbf{\textit{#1}}}}
\newcommand{\AttributeTok}[1]{\textcolor[rgb]{0.77,0.63,0.00}{#1}}
\newcommand{\BaseNTok}[1]{\textcolor[rgb]{0.00,0.00,0.81}{#1}}
\newcommand{\BuiltInTok}[1]{#1}
\newcommand{\CharTok}[1]{\textcolor[rgb]{0.31,0.60,0.02}{#1}}
\newcommand{\CommentTok}[1]{\textcolor[rgb]{0.56,0.35,0.01}{\textit{#1}}}
\newcommand{\CommentVarTok}[1]{\textcolor[rgb]{0.56,0.35,0.01}{\textbf{\textit{#1}}}}
\newcommand{\ConstantTok}[1]{\textcolor[rgb]{0.00,0.00,0.00}{#1}}
\newcommand{\ControlFlowTok}[1]{\textcolor[rgb]{0.13,0.29,0.53}{\textbf{#1}}}
\newcommand{\DataTypeTok}[1]{\textcolor[rgb]{0.13,0.29,0.53}{#1}}
\newcommand{\DecValTok}[1]{\textcolor[rgb]{0.00,0.00,0.81}{#1}}
\newcommand{\DocumentationTok}[1]{\textcolor[rgb]{0.56,0.35,0.01}{\textbf{\textit{#1}}}}
\newcommand{\ErrorTok}[1]{\textcolor[rgb]{0.64,0.00,0.00}{\textbf{#1}}}
\newcommand{\ExtensionTok}[1]{#1}
\newcommand{\FloatTok}[1]{\textcolor[rgb]{0.00,0.00,0.81}{#1}}
\newcommand{\FunctionTok}[1]{\textcolor[rgb]{0.00,0.00,0.00}{#1}}
\newcommand{\ImportTok}[1]{#1}
\newcommand{\InformationTok}[1]{\textcolor[rgb]{0.56,0.35,0.01}{\textbf{\textit{#1}}}}
\newcommand{\KeywordTok}[1]{\textcolor[rgb]{0.13,0.29,0.53}{\textbf{#1}}}
\newcommand{\NormalTok}[1]{#1}
\newcommand{\OperatorTok}[1]{\textcolor[rgb]{0.81,0.36,0.00}{\textbf{#1}}}
\newcommand{\OtherTok}[1]{\textcolor[rgb]{0.56,0.35,0.01}{#1}}
\newcommand{\PreprocessorTok}[1]{\textcolor[rgb]{0.56,0.35,0.01}{\textit{#1}}}
\newcommand{\RegionMarkerTok}[1]{#1}
\newcommand{\SpecialCharTok}[1]{\textcolor[rgb]{0.00,0.00,0.00}{#1}}
\newcommand{\SpecialStringTok}[1]{\textcolor[rgb]{0.31,0.60,0.02}{#1}}
\newcommand{\StringTok}[1]{\textcolor[rgb]{0.31,0.60,0.02}{#1}}
\newcommand{\VariableTok}[1]{\textcolor[rgb]{0.00,0.00,0.00}{#1}}
\newcommand{\VerbatimStringTok}[1]{\textcolor[rgb]{0.31,0.60,0.02}{#1}}
\newcommand{\WarningTok}[1]{\textcolor[rgb]{0.56,0.35,0.01}{\textbf{\textit{#1}}}}
\usepackage{longtable,booktabs}
\usepackage{calc} % for calculating minipage widths
% Correct order of tables after \paragraph or \subparagraph
\usepackage{etoolbox}
\makeatletter
\patchcmd\longtable{\par}{\if@noskipsec\mbox{}\fi\par}{}{}
\makeatother
% Allow footnotes in longtable head/foot
\IfFileExists{footnotehyper.sty}{\usepackage{footnotehyper}}{\usepackage{footnote}}
\makesavenoteenv{longtable}
\usepackage{graphicx}
\makeatletter
\def\maxwidth{\ifdim\Gin@nat@width>\linewidth\linewidth\else\Gin@nat@width\fi}
\def\maxheight{\ifdim\Gin@nat@height>\textheight\textheight\else\Gin@nat@height\fi}
\makeatother
% Scale images if necessary, so that they will not overflow the page
% margins by default, and it is still possible to overwrite the defaults
% using explicit options in \includegraphics[width, height, ...]{}
\setkeys{Gin}{width=\maxwidth,height=\maxheight,keepaspectratio}
% Set default figure placement to htbp
\makeatletter
\def\fps@figure{htbp}
\makeatother
\setlength{\emergencystretch}{3em} % prevent overfull lines
\providecommand{\tightlist}{%
  \setlength{\itemsep}{0pt}\setlength{\parskip}{0pt}}
\setcounter{secnumdepth}{-\maxdimen} % remove section numbering
\ifluatex
  \usepackage{selnolig}  % disable illegal ligatures
\fi

\title{named entity recognition}
\author{Selim Ben Ismail}
\date{7/3/2022}

\begin{document}
\maketitle

\begin{Shaded}
\begin{Highlighting}[]
\FunctionTok{heatmap}\NormalTok{(m\_distance,}\AttributeTok{Rowv =} \ConstantTok{NA}\NormalTok{, }\AttributeTok{Colv =} \ConstantTok{NA}\NormalTok{)}
\end{Highlighting}
\end{Shaded}

\includegraphics{ren_files/figure-latex/unnamed-chunk-2-1.pdf}

\begin{Shaded}
\begin{Highlighting}[]
\FunctionTok{kable}\NormalTok{(df\_closeDistances, }\AttributeTok{caption =} \StringTok{"Occurences proches"}\NormalTok{)}
\end{Highlighting}
\end{Shaded}

\begin{longtable}[]{@{}lll@{}}
\caption{Occurences proches}\tabularnewline
\toprule
Antrhoponyme\_1 & Antrhoponyme\_2 & Distance\tabularnewline
\midrule
\endfirsthead
\toprule
Antrhoponyme\_1 & Antrhoponyme\_2 & Distance\tabularnewline
\midrule
\endhead
SYMON DE PIERONNE & JAKEMON DE PIERONNE & 4\tabularnewline
SYMON BELOT & SYMON BUEE & 4\tabularnewline
JEHAN MOUTON & JEHAN MULET & 4\tabularnewline
JEHAN PILATE & JEHAN MULET & 4\tabularnewline
MARIEN DE SYM & MARIIEN DE SYM & 1\tabularnewline
MARIIEN DE SYM & MARIEN DE SYM & 1\tabularnewline
JAKEMON LE VAKIER & JAKEMON LE PLAKEUR & 4\tabularnewline
SYMON BUEE & SYMON BELOT & 4\tabularnewline
JEHAN LESCUTIER & JEHAN LE CARLIER & 4\tabularnewline
JEHAN LE PIGNIER & JEHAN LE CARLIER & 4\tabularnewline
JEHAN LE CARLIER & JEHAN LESCUTIER & 4\tabularnewline
JEHAN LE CARLIER & JEHAN LE PIGNIER & 4\tabularnewline
PIERON RAMET & PIERON LE RAMET & 3\tabularnewline
ROBIERT D'ESTREES & ROBERT D'ESTREES & 1\tabularnewline
PIERON LE RAMET & PIERON RAMET & 3\tabularnewline
PIERON FAUKE & PIERON RAMET & 4\tabularnewline
JEHAN MULET & JEHAN MOUTON & 4\tabularnewline
JAKEMON DE PIERONNE & SYMON DE PIERONNE & 4\tabularnewline
ROBIERT CERFUEL & ROBIERT CIERFUEL & 1\tabularnewline
JEHAN MULET & JEHAN PILATE & 4\tabularnewline
SAINTAIN DE DICHI & MARTIN DE DICHI & 4\tabularnewline
JAKEMON LE PLAKEUR & JAKEMON LE VAKIER & 4\tabularnewline
HENNIN CHASSE DIEU & WERIN CHASE DIEU & 4\tabularnewline
MARTIN DE DICHI & SAINTAIN DE DICHI & 4\tabularnewline
ROBERT D'ESTREES & ROBIERT D'ESTREES & 1\tabularnewline
PIERON RAMET & PIERON FAUKE & 4\tabularnewline
MAROTAIN PAINMOULLIET & MARIIEN PAINMOULLIET & 4\tabularnewline
WERIN CHASE DIEU & HENNIN CHASSE DIEU & 4\tabularnewline
ROBIERT CIERFUEL & ROBIERT CERFUEL & 1\tabularnewline
MARIIEN PAINMOULLIET & MAROTAIN PAINMOULLIET & 4\tabularnewline
MARIEN LE SOURDE & MARIIEN LE SOURDE & 1\tabularnewline
MARIIEN LE SOURDE & MARIEN LE SOURDE & 1\tabularnewline
\bottomrule
\end{longtable}

\begin{Shaded}
\begin{Highlighting}[]
\FunctionTok{print}\NormalTok{(}\StringTok{"}\SpecialCharTok{\textbackslash{}n}\StringTok{ nombre de suspicion de doublons : "}\NormalTok{)}
\end{Highlighting}
\end{Shaded}

{[}1{]} ``\n nombre de suspicion de doublons :''

\begin{Shaded}
\begin{Highlighting}[]
\FunctionTok{length}\NormalTok{(}\FunctionTok{unique}\NormalTok{(df\_closeDistances}\SpecialCharTok{$}\NormalTok{Antrhoponyme\_1))}
\end{Highlighting}
\end{Shaded}

{[}1{]} 29

\begin{Shaded}
\begin{Highlighting}[]
\FunctionTok{print}\NormalTok{(}\StringTok{"}\SpecialCharTok{\textbackslash{}n}\StringTok{ nombre à 1 de distance : "}\NormalTok{)}
\end{Highlighting}
\end{Shaded}

{[}1{]} ``\n nombre à 1 de distance :''

\begin{Shaded}
\begin{Highlighting}[]
\FunctionTok{nrow}\NormalTok{(df\_closeDistances[df\_closeDistances}\SpecialCharTok{$}\NormalTok{Distance }\SpecialCharTok{==} \DecValTok{1}\NormalTok{, ])}
\end{Highlighting}
\end{Shaded}

{[}1{]} 8

\begin{Shaded}
\begin{Highlighting}[]
\FunctionTok{print}\NormalTok{(}\StringTok{"}\SpecialCharTok{\textbackslash{}n}\StringTok{ nombre à 2 de distance : "}\NormalTok{)}
\end{Highlighting}
\end{Shaded}

{[}1{]} ``\n nombre à 2 de distance :''

\begin{Shaded}
\begin{Highlighting}[]
\FunctionTok{nrow}\NormalTok{(df\_closeDistances[df\_closeDistances}\SpecialCharTok{$}\NormalTok{Distance }\SpecialCharTok{==} \DecValTok{2}\NormalTok{, ])}
\end{Highlighting}
\end{Shaded}

{[}1{]} 0

\begin{Shaded}
\begin{Highlighting}[]
\FunctionTok{print}\NormalTok{(}\StringTok{"}\SpecialCharTok{\textbackslash{}n}\StringTok{ nombre à 3 de distance : "}\NormalTok{)}
\end{Highlighting}
\end{Shaded}

{[}1{]} ``\n nombre à 3 de distance :''

\begin{Shaded}
\begin{Highlighting}[]
\FunctionTok{nrow}\NormalTok{(df\_closeDistances[df\_closeDistances}\SpecialCharTok{$}\NormalTok{Distance }\SpecialCharTok{==} \DecValTok{3}\NormalTok{, ])}
\end{Highlighting}
\end{Shaded}

{[}1{]} 2

\begin{Shaded}
\begin{Highlighting}[]
\FunctionTok{print}\NormalTok{(}\StringTok{"}\SpecialCharTok{\textbackslash{}n}\StringTok{ nombre à 4 de distance : "}\NormalTok{)}
\end{Highlighting}
\end{Shaded}

{[}1{]} ``\n nombre à 4 de distance :''

\begin{Shaded}
\begin{Highlighting}[]
\FunctionTok{nrow}\NormalTok{(df\_closeDistances[df\_closeDistances}\SpecialCharTok{$}\NormalTok{Distance }\SpecialCharTok{==} \DecValTok{4}\NormalTok{, ])}
\end{Highlighting}
\end{Shaded}

{[}1{]} 22

\end{document}
